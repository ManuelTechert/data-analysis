% !TEX root =  master.tex 
\chapter{Projektaufbau}


\section{Anforderungen an dieses Projekt -- Fabian Pulch}
Diesem Projekt liegt eine Aufgabenstellung des Dozenten Gregor Tielsch zugrunde. Demnach soll ein Computerspiel entwickelt werden. Bei diesem Computerspiel soll es sich um ein computergestütztes Unternehmensplanspiel handeln. Dieses Planspiel soll modellhaft Unternehmensprozesse eines Unternehmens beinhalten, welches auf einem Oligopolmarkt agiert und konkurriert. Ein typischer Ablauf eines Planspiels mit einer Informationsausgabe, Steuerungsphase, Entscheidungs- und einer Simulationsphase soll abgebildet werden.
%Diagramm Folien Tielsch
\subsection{Grundgedanken}
Um diesen Grundanforderungen gerecht zu werden, wurden zwei Beschlüsse gefasst, welche im folgenden erläutert werden. Das Spiel, dessen Analyse, Entwurf und Implementierung im Rahmen dieses Projektes in Angriff genommen wurde, soll mehrspielerfähig sein. Mehrere Spieler sollen also in der Lage sein, in einer Konkurrenzsituation zueinander zu stehen und somit durch ihre Handlungen einen unmittelbaren Einfluss auf den Markt mit anderen Marktteilnehmern ausüben. Um eine solche Mehrspielerfähigkeit zu gewährleisten, hat man sich gegen ein sogenanntes ''Hot Seat'' Verfahren entschieden, da heutzutage weitläufig diverse Endgeräte verfügbar sind. Eine Teilung der Ressource ist dadurch in den meisten Fällen nicht mehr notwendig, d.h. es ist angedacht, dass sich Spieler mit verschiedenen Endgeräten auf einen Server verbinden. Dieser Server verwaltet das Spielgeschehen. Er empfängt Eingabedaten und sendet Ereignisse bzw. Meldungen zum Spielverlauf zurück.
%Zeichnung Mulitplayer Mensch-Client-Server

Des Weiteren soll das Spiel, welches im Rahmen dieser Arbeit geschaffen wird, eine Dienstleistungs- oder Industriebetriebsumgebung beinhalten, welche einerseits nicht allzu komplex ist, um in der gegebenen Zeit modellhaft abgebildet zu werden und dennoch umfangreich genug, sodass eine spätere Erweiterung der Spielmechanik möglich ist. Um diese Problematik zu lösen, wurden drei Spielfelder bzw. Unternehmensszenarien unterschiedlicher Branchen mit ihren Problematiken und Möglichkeiten ausgearbeitet und gegenüber gestellt. Die erste Möglichkeit war die Abbildung eines Logistik und Transportunternehmens innerhalb des Planspiels. Der Spieler könnte eine Transportflotte, bestehend aus unterschiedlichsten Arten von Fahrzeugen, verwalten und durch ein gezieltes Management beispielsweise durch den Bau von neuen Logistikzentren expandieren und neue Bereiche freischalten.

Die zweite Lösung, welche ausführlich debattiert wurde, war ein Industrieunternehmen, welches für den Endverbraucher, einfache Produkte, wie diverse Möbel, herstellt. Erweiterbarkeit ist im diesem Szenario durch die Einführung neuer Produkte möglich oder strategischen Entscheidungen z.B. eine Unterscheidung zwischen einer Differenzierungs- und einer Kostenführerschaftsstrategie.

Das letzte Szenario, welches in dieser frühen Entscheidungsphase grob skizziert wurde, ist ein Planspiel mit dem Fokus auf die Baubranche. Der Spieler agiert in der Rolle als Geschäftsführer eines Bauunternehmens und muss vorhandene Ressourcen möglichst effizient managen, um eingehende Aufträge annehmen zu können. Durch das erfolgreiche Abschließen angenommener Auftrage, erhält der Spieler Geld, wodurch er seine Bilanz verbessert und Möglichkeiten zur Expansion nutzen kann, beispielsweise den Kauf weiterer Maschinen, um mehr oder größere Aufträge annehmen zu können. 

Nach einen ausführlichen Diskurs wurde entschieden, dass dieses Projekt das Ziel verfolgt im zu erstellenden Planspiel das dritte Szenario abzubilden. Es bietet vielseitige Erweiterungsmöglichkeiten und kann dennoch so weit wie nötig abstrahiert werden. Im Vergleich zu den ersten zwei Ansätzen überzeugt vor allem die umfangreiche Abbildungsmöglichkeit von Abhängigkeiten wie Umwelteinflüssen oder Lieferschwierigkeiten auf einzelne Aufträge und deren Erfolg für den Spieler. Dadurch wird ein Spielerlebnis geschaffen, welches dem Spieler gegenüber stets neue Facetten aufzeigt. So wird gewährleistet, dass das Spiel auch bei mehrmaligem Spielen hintereinander nicht allzu monoton wirkt.

Die folgenden Kapitel erläutern den weiteren Entwurf, die teilweise Implementierung des Konzeptes, sowie die aufgetretenen Problemstellungen und entwickelten Lösungen.

%Hier könnte so etwas hin wie, mit welchen Tools arbeiten wir und wie organisieren wir uns.
\section{Organisation des Projektes -- Daniel Pies}

\subsection{Codeverwaltung mit GitHub}

Um alle gemeinsam an einem Programm arbeiten zu können und den gleichen Code zu nutzen, haben wir unseren Code in einem GitHub-Projekt organisiert. Github ermöglicht es, dass mehrere Leute am selben Programm zur gleichen Zeit arbeiten können. GitHub erkennt alle Änderungen, die ein Entwickler an einem Projekt vornimmt und kennzeichnet diese. So kann jeder andere Entwickler sehen, was von wem wann wie geändert wurde. So konnte beispielsweise Person 1 die berecheAbschreibungen() Methode in der Klasse Unternehmen schreiben, während Person 2 gleichzeitig an der Klasse Baumaschinen entwickelt hat. Es sollten nur nicht zwei Personen gleichzeitig in der gleichen Klasse bzw. Datei arbeiten, da dies beim Hochladen der Änderungen zu Problemen kommen kann. Wenn zwei Personen an der gleichen Datei arbeiten, kann Github nicht erkennen, ob jetzt die Änderungen von Person 1 oder die von Person 2 verwendende werden sollen. Man muss sich also abstimmen, wer an was programmiert, um diese Fehler zu vermeiden.

%\subsection{Qualitätssicherung mit SonarCube und Jenkins}


\subsection{Kommunikationskanäle}

Zur Kommunikation untereinander haben wir eine Whatsapp-Gruppe, in der jeder kurze Stati posten konnte oder auch schreiben konnte, in welcher Klasse er gerade Programmiert, damit es nicht zu Problemen beim Hochladen in GitHub kommt, wie im vorherigen Absatz beschrieben.

Des Weiteren nutzen wir einen Discord-Server. Discord ermöglicht einen Gruppen-Voice-Chat, um auch abends oder am Wochenende miteinander sprechen zu können und sich unkompliziert über das Projekt, die aktuelle Programmierung oder das nächste Treffen in der Uni abstimmen zu können.


\subsection{Dokumentation mit LaTeX}

Zur Dokumentation des Projekts und den damit verbundenen Schreiben dieser Seminararbeit verwenden wir das Text-Satz Programm LaTeX. Auch dieses LaTeX-Dokument haben wir mit GitHub verwaltet, da dann jeder daran arbeiten kann, ohne immer den aktuellsten Stand des Dokuments per Mail oder ähnlichem verteilen zu müssen. Außerdem ermöglicht LaTeX es, dass man für jedes Kapitel eine eigene Datei anlegen kann und dieser in einer Master-Datei zusammenfügen kann. Dadurch kann jeder in seinen Kapiteln  arbeiten und diese bei GitHub hochladen, ohne dass es zu Upload-Konflikten kommt.