% !TEX root =  master.tex 
\chapter{Fazit und Reflexion des Projekts}

Zu Beginn stand die Planungsphase, in der wir uns für ein Planspiel der Baubranche entschieden haben. Nach der Aufgabenverteilung ging es für uns danach mit großen Plänen an die Arbeit. Leider mussten wir kurzfristig unsere Ziele etwas reduzieren, da wir zwei miteingeplante Mitglieder unserer Gruppe verloren hatten.
Als dies bekannt wurde, mussten wir uns und unsere Ziele neu organisieren. Wichtig war uns vor allem das Fachkonzept, welches eine möglichst realitätsnahe Simulation ermöglichen soll, die Benutzeroberfläche hielten wir pragmatisch und funktional.

Zusammenfassend lässt sich das Projekt als eine sehr interessante und lehrreiche Erfahrung darstellen. Wir haben vor allem gelernt, als Team zusammen zu arbeiten und uns gemeinsam zu organisieren. Die Arbeitsatmosphäre war sehr entspannt und zielorientiert, die Kommunikation erfolgte dank Discord, Whatsapp und Co problemlos.
Auch beim Erstellen des Spiels selber, konnten wir viel dazu lernen. Um zu unserem Ergebnis zu kommen, mussten wir nahezu alle bisher kennengelernten Programmierfähigkeiten aus dem Studium anwenden, kombinieren, ausbauen oder uns in noch unbekannte Themen wie das Testen von Software einlesen. Wichtig war hierbei auch der Umgang mit Rückschlägen beim Ausprobieren neuer Methoden. 

Mit unserem Ergebnis sind wir rückblickend sehr zufrieden. Wir haben ein Spiel erstellt, welches realitätsnah auf viele Umwelt- und Wirtschaftsfaktoren eingeht und dem Spieler so einen Einblick in die Führung eines Unternehmens mit allen notwendigen Aktionen gibt.
Gerne hätten wir noch die Benutzeroberfläche etwas ausgebaut, wäre noch Zeit offen gewesen.








