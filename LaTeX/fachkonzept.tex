% !TEX root =  master.tex 
\chapter{Fachkonzept und Programmierung -- Nico Popiolek}

\section{Die Klasse Unternehmen}
Kernstück des Fachkonzepts bildet die Klasse Unternehmen, welche alle relevanten Daten, Eigenschaften und Funktionen des Unternehmens beinhaltet. Dazu zählen sowohl Anlage- wie Umlaufvermögen, als auch virtuelle Werte der Spielmechanik wie eine Statistik oder der Einfluss des Marketings. Zwischen jedem Unternehmen und Spieler besteht eine 1 zu 1 Beziehung, es ist also nicht möglich für einen Spieler, mehrere Unternehmen in der gleichen Runde zu verwalten und ebenso gibt es keine Unternehmen, welche 'autonom' bzw. von der Spielmechanik gesteuert am Spiel teilnehmen.

Es wurden die folgenden Attribute eines Unternehmens in das Fachkonzept integriert:

\begin{itemize}
	 
\item Immobilien, leider konnte die Klasse jedoch aufgrund Zeitmangels nicht vollständig implementiert werden. Das Fachkonzept sah vor, dass ein Unternehmen ab einer bestimmten Anzahl Mitarbeiter und Lagerbestände weitere oder größere Immobilien erwerben muss, um den Betrieb aufrecht zu erhalten. So erhält der Spieler lediglich eine Immobilie mit einem Wert von 1.600.000\euro zu Beginn des Spiels, welche dann mit jeder Runde abgeschrieben wird bis auf einen Wert von einem Euro, sonst jedoch keine weitere Funktion erfüllt.

\item Baumaterialien, welche in Form von Stahl, Beton und Holz gehandelt und verwendet werden können. Dafür wurde eine ArrayList implementiert, welche die drei Rohstofftypen als Objekte beinhaltet. Bei Kauf oder Verwendung der Materialien wird jeweils die Variable \texttt{anzahl} verändert. Es stehen die Materialien Holz, Beton und Stahl zu Verfügung, welche alle in Tonnen gehandelt werden. Als grober Richtwert für ein mehrstöckiges Einfamilienhaus wurden 21t Holz, 10t Stahl und 100t Beton angenommen, deren Nettopreise in Deutschland aktuell bei etwa 100\euro, 500\euro und 30\euro  liegen.

\item Baumaschinen, welche als Bagger, Kräne und LKWs gehandelt und auf den Baustellen eingesetzt werden können. Anders als bei den Baumaterialien existiert hier eine ArrayList, welche jede einzelne Baumaschine als individuelles Objekt beinhaltet. Diese Vorgehensweise war nötig, da Maschinen funktional komplexer sind als Baumaterialien. So erhält jede Maschine ein Attribut \texttt{inVerwendung}, welches kennzeichnet, ob die Maschine gerade auf einer Baustelle im Einsatz ist, oder für neue Aufträge zur Verfügung steht. Des weiteren hat jede Baumaschine einen Wert, welcher zu Anfang dem Kaufpreis entspricht und dann jährlich um 1/8 abgeschrieben wird. Die günstigste Baumaschine ist der Lkw mit einem Neupreis von 80.000\euro, danach folgt der Bagger mit 150.000\euro und am teuersten ist der Kran mit 170.000\euro. Nach Ablauf der acht Jahre, verbleiben die Maschinen weiter mit dem Wert von einem Euro als Anlagevermögen in der Bilanz.

\item Wertpapierdepots, welche dem Spieler die Möglichkeit geben, in Aktien oder Gold zu investieren. Die Entwicklung des Aktienmarkts orientiert sich dabei an jener der Gesamtwirtschaft und führt zu Kursveränderungen und Dividendenausschüttungen an den Spieler. So erhält das Unternehmen bei einem Wert unter 50 keinerlei Dividende seiner Aktien und muss Kursverluste hinnehmen, bei einer guten Wirtschaftslage über 50 steigt der Wert der Aktien und eine jährliche Dividende, die vom aktuellen Marktscore abhängt, wird ausgezahlt. Umgekehrt verhält es sich mit Investitionen in Gold. Es wird angenommen, dass Anleger bei schlechter Wirtschaftslage und fallenden Aktienkursen verstärkt in Gold investieren, was zu einer Kurssteigerung führt und umgekehrt. So kann der Spieler sich absichern, falls er einen größeren Wirtschaftscrash befürchtet. Bei einer langsam und kontinuierlich wachsenden Wirtschaft profitieren sowohl der Aktienmarkt als auch der Kurs von Gold. Eine Dividende auf Gold wird nicht ausgezahlt.

\item Kredite werden grundsätzlich in Form von Tilgungsdarlehen an die Spieler ausgegeben. Die jährlichen Rückzahlungen beinhalten sowohl den gleichbleibenden Tilgungsanteil, welcher sich aus der Kreditsumme und der Laufzeit ergibt als auch den Zinsanteil, welcher jährlich neu berechnet und auf 5\% der verbleibenden Kreditsumme gesetzt wird. Die Summe und Laufzeit kann der Spieler selbst auswählen, eine vorzeitige Tilgung des Kredits ist jedoch nicht vorgesehen.

\item Bauarbeiter, wobei hier unterschieden wird zwischen solchen, die aktuell auf einer Baustelle aktiv sind und freien Bauarbeitern, welche für eingehende Angebote zur Verfügung stehen. Inklusive Lohnnebenkosten fallen pro Bauarbeiter Kosten in Höhe von 50.000 Euro pro Jahr an. Für das Entlassen eines Bauarbeiters müssen unabhängig von seinem Status 15.000\euro Abfindung gezahlt werden. Hat ein Bauarbeiter auf einer Baustelle einen tragischen Arbeitsunfall, kann er zusätzlich Schmerzensgeld verlangen, im Falle einer permanenten Invalidität oder sogar Tod des Angestellten, werden hohe Schmerzensgeldzahlungen fällig und die Anzahl der Bauarbeiter wird um eins verringert.

\item Verwaltungsangestellte sind bei wachsender Größe des Unternehmens zunehmend notwendig und verursachen Kosten von 55.000 Euro pro Mitarbeiter und Jahr. Soll ein Verwaltungsangestellter entlassen werden, so muss das Unternehmen ihm 18.000\euro Abfindung zahlen. Bis zum Zeitpunkt der Fertigstellung dieses Projekts konnten leider keine weitere den Spielablauf beeinflussenden Funktionen für Verwaltungsangestellte implementiert werden. Ursprünglich war angedacht, dass das Unternehmen bei wachsender Größe auch mehr Verwaltungsangestellte benötigt und diese gegebenenfalls auch weitere spielerische Vorteile mit sich bringen, wie beispielsweise erfolgreichere Marketingkampagnen.

\item Bankkonto, welches die flüssigen Geldmittel des Unternehmens repräsentiert und somit den zentralen Knotenpunkt aller Geldflüsse darstellt. Alle Zu- und Abflüsse von Geld werden über das Bankkonto dargestellt, Kredite werden also beispielsweise direkt von der Bank auf das Konto des Unternehmens ausgezahlt und erst im zweiten Schritt für Investitionen verwendet. Gibt der Spieler mehr Geld aus, als auf dem Konto vorhanden ist, so rutscht dies zunächst im UI ins Minus. Wird die nächste Runde gestartet, so findet eine Prüfung auf den Kontostand statt und sollte dieser negativ sein, so wird automatisch ein Kredit über den Fehlbetrag aufgenommen, sodass der Spieler kein zinsloses Darlehen durch einfaches Geldausgeben erhält.

\item Marketing, welches dem Spieler unterschiedliche Formen des Marketings zu Verfügung stellt, die sich in Kapitalintensivität und Effektivität unterschieden und dem Spieler in Form eines Marketing-Scores einen Vorteil gegenüber anderen Spielern verschaffen können. Außerdem kann der Spieler seinen eigenen Marketingscore abfragen, wobei der relative Marketingscore Auswirkungen auf die eingehenden Aufträge des Spielers hat. Bei jeder Investition eines Spielers erhöht sich sowohl der eigene als auch der globale Marketingscore, welcher die Summer der Marketingscores aller Spieler ist, um den gleichen Wert. Damit steigt der eigene relative Anteil, wogegen der aller anderen Spieler gleichzeitig sinkt. Dieser Mechanismus soll einerseits die Konkurrenz unter den Spielern darstellen und ein Gegenseitiges Verdrängen ermöglichen und andererseits sicherstellen, dass die Nachfrage im Markt nicht von der Werbung der Unternehmen, sondern vielmehr von der wirtschaftlichen Lage, also dem verfügbaren Kapital der Kunden abhängt - ein erfolgreiches Marketing sichert dem Spieler lediglich ein größeres Stück des nach wie vor gleich großen Kuchens zu. Ein hoher relativer Marktanteil bedeutet dennoch signifikante Vorteile gegenüber den anderen Mitspielern.

\end{itemize}

Es wird davon ausgegangen, dass jeder Spieler zu Beginn des Spiels mit einem sehr kleinen oder neuen Unternehmen, aber großer Kapitaldecke startet. Deshalb wird jedes neue Unternehmen im Konstruktor mit einer Immobilie, einem kleinen Aktiendepot, ein paar Angestellten, Rohstoffen für eine kleine bis mittlere Baustelle und je einer Baumaschine jedes Typs sowie 5 Millionen Euro auf dem Konto initialisiert. So sieht der Spieler gleich zur ersten Runde was mit welchen Positionen im UI passiert und kann durch die große Menge liquider Mittel das Unternehmen nach seinen Präferenzen aufbauen und bereits früh entscheiden, ob er beispielsweise auf einer Verdrängungsstrategie der anderen durch große Investitionen in Marketing setzt, auf Wertpapiere als zweites Standbein oder schnelles Wachstum durch Kauf von Material und Maschinen und Annahme möglichst großer Aufträge.


\section{Steuerung und Spielablauf}
Die teilnehmenden Unternehmen werden verwaltet über die Klasse Unternehmensverwaltung. Die Unternehmensverwaltung ist ein Singleton und enthält eine ArrayList mit allen Unternehmen und stellt Methoden bereit, mit denen die Funktionen der Unternehmen ausgeführt werden können. Analog dazu wurden auch die Angebotsverwaltung sowie der GameEventHandler und GameEventGenerator als Singletons implementiert.
Die Angebotsverwaltung dient dazu, neue Angebote zu generieren, anzunehmen, abzulehnen und ihre Dauer zu verwalten. Zusätzlich werden hier Events erzeugt, die sich auf bereits angenommene Angebote auswirken, wie zum Beispiel eine Verlängerung der Baustelle durch bestimmte Ereignisse.
Jeder Spieler erhält pro Runde drei Angebote. Diese unterliegen sowohl in ihrer Größe als auch ihrer Rentabilität Zufallswahrscheinlichkeiten, die sich in vorgegebenen Korridoren bewegen und durch die aktuelle Konjunktur sowie den relativen Marketingscore des Unternehmens beeinflusst werden. Zunächst wird für jedes Angebot eine von drei möglichen Größen 1 bis 3 zufällig ermittelt. Ein Angebot der Kategorie 1 entspricht dem kleinstmöglichen Angebot und orientiert sich an  dem Umfang eines normalen Einfamilienhauses. Kategorie 2 entspricht einer ungefähr doppelt so großen Baustelle und Kategorie verlangt noch einmal etwa den dreifachen Materialeinsatz.
Entsprechend der Größe wird nun für alle benötigten Ressourcen und auch den Preis in definierten Grenzen Zufallswerte ermittelt und schließlich um den Konjunktur- und Marketingmodifikator verändert. Hat ein Unternehmen einen besonders hohen relativen Marketingscore, sind die Angebote wesentlich lukrativer als die der Konkurrenz. Das fördert einerseits den Anreiz, mehr Geld in Marketing zu investieren, um so die anderen Spieler auszustechen und bessere Angebote zu erhalten und andererseits zwingt es die Spieler dazu, jedes Angebot einzeln genau zu prüfen, denn unabhängig von der Größe und Marktsituation kann ein Angebot für den gegebenen Preis einen höheren oder niedrigeren Material- und Arbeitseinsatz verlangen und somit mehr oder weniger lukrativ sein. Insbesondere die Lohnkosten sind hier als entscheidender Faktor zu berücksichtigen, zusätzlich aber auch Kosten für unvorhergesehene Ereignisse wie ein Wasserrohrbruch mit einzukalkulieren. Alle Grenzen wurden jedoch so gesetzt, dass kein Angebot von vornherein einen sicheren Verlust erzielt, da ein reales Unternehmen einen solchen Auftrag nicht annehmen würde. Angebote können grundsätzlich nur angenommen werden, wenn alle benötigten Rohstoffe, Baumaschinen und Bauarbeiter zu Beginn an vorhanden sind. Zuführen weiterer Ressourcen während der Bauzeit ist nicht möglich und da eine Runde etwa einem verstrichenen Jahr entspricht, sind alle Angebote nur genau eine Runde lang gültig. Entscheidet sich der Spieler ein Angebot nicht anzunehmen, wird es nicht wieder auftauchen und stattdessen durch neue Angebote ersetzt.

Durch ein oder mehrere unvorhergesehene Events kann eine Baustelle jedoch zum Verlustgeschäft werden - auch hier muss der Spieler planen und abwägen, ob zum gegebenen Preis ausreichend Puffer vorhanden ist.
Nimmt der Spieler ein Angebot an, werden ihm am Ende der Runde alle benötigten Rohstoffe abgezogen und die Anzahl benötigter Bauarbeiter und Maschinen stehen für die Dauer der Baustelle nicht zur Verfügung. Während der Bauphase werden zusätzlich in jeder Runde Wahrscheinlichkeiten für diverse Zufallsevents auf den Baustellen generiert. Dies reicht von einem Wasserschaden, welcher lediglich einen geringen finanziellen Verlust für den Spieler bedeutet bis hin zu Events wie dem Auffinden geschützter Tierarten auf dem Baugrundstück, was eine Verlängerung der Bauzeit um ein Jahr bedeutet und für den Spieler hohe Lohnkosten und den Verzicht auf Bauarbeiter und Maschinen nach sich zieht. Ebenfalls möglich sind Arbeitsunfälle, welche zu Schmerzensgeldzahlungen und im schlimmsten Fall sogar zum permanenten Verlust eines Bauarbeiters führen können. Sämtliche Events werden dem Spieler im UI in Textform dargestellt.
Erreicht die verbleibende Bauzeit null, werden die Bauarbeiter und Maschinen wieder verfügbar für neue Aufträge und der gesamte Preis des Angebots wird dem Konto gutgeschrieben. Es findet keine An- oder Ratenzahlung statt.

Der GameEventGenerator stellt die Methode \texttt{generiereWahrscheinlichkeit} zur Verfügung, welche an vielen anderen Stellen verwendet wird, der Handler ist dafür zuständig, die globalen Events zu verwalten und beispielsweise das Wirtschaftsklima jede Runde zu errechnen. Aufgrund Zeitmangels konnten hier neben der jährlichen Veränderung des Wirtschaftsklimas keine weiteren Events implementiert werden. Ideen waren beispielsweise ein branchenweiter Streik der Bauarbeiter welcher alle Spieler betrifft und längere Bauzeiten sowie höhere Löhne zur Folge hat, ein Börsencrash, Zinsveränderungen und schwankende Rohstoffpreise.

Die Klasse \texttt{Spiel}, welche den Spielablauf nach jeder Runde steuert, greift in der zentralen Methode \texttt{starteRunde} auf die Singletons zu und führt so in logischer Reihenfolge die nötigen Operationen zwischen den einzelnen Runden aus. Dazu zählen die Abrechnung der Gehälter, Abschreibungen, Aufnahme und Rückzahlung der Kredite, die Berechnung der neuen Aktienkurse mit der zugehörigen Dividendenausschüttung und zum Schluss die Berechnung des neuen Eigenkapitals für alle teilnehmenden Unternehmen, welches in der Bilanz verwendet wird. Während der Programmierung war es außerdem Konvention, alle Exceptions einheitlich in der \texttt{Spiel}-Klasse zu behandeln um die Übersicht und Wartbarkeit des Projekts zu verbessern.

Die Eingabewerte unterteilen sich in solche, die direkt nach ihrer Eingabe verwaltet werden - dazu zählt die Manipulation der Mitarbeiter, Maschinen, Baumaterial und Kredite - und solche, die erst nach Ende der Runde verarbeitet werden. Diese Entscheidung wurde getroffen, damit der Spieler während einer Runde auf die ihm gegebenen Angebote reagieren kann und gegebenenfalls Material oder Maschinen einkaufen kann, um einen besonders lukrativen Auftrag annehmen zu können, welcher andernfalls nach Ende der Runde nicht mehr zur Verfügung stünde, da alle Angebote neu generiert wurden.

Nachdem alle Berechnungen abgeschlossen wurden, wird nun an dieser Stelle der Rundenzähler erhöht und zum Spielende der Gewinner ermittelt.



